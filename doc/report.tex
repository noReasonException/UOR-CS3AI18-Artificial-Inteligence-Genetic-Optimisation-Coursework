\documentclass[openany]{article}


%Standard Stefanos Packages
\usepackage[utf8]{inputenc}
\usepackage{dirtytalk}
\usepackage{amsmath}
\usepackage{mathtools}  
\mathtoolsset{showonlyrefs} 
\usepackage{graphicx}
\usepackage{mdframed}
\usepackage{lipsum}
\usepackage{cancel}
\usepackage{systeme}
\usepackage{pgfplots}
\usepackage{textcomp}
\usepackage{amssymb}
\usepackage{geometry}
\usepackage{tikz-cd}
\usetikzlibrary{arrows}
\geometry{a4paper}
\graphicspath{ {./res/} }
\usepackage{float}
\restylefloat{table}
\newcommand{\comment}[1]{%
	\text{\phantom{(#1)}} \tag{#1}
}
 \title{\line(3,0){250}\\Artificial Intelligence \\ Genetic Algorithms Coursework  \\\line(3,0){250}}
\usepackage{pgfplots}
\newmdtheoremenv{note}{Note}
\pgfplotsset{compat=1.17}

\newmdtheoremenv{definition}{Definition}

%Extra Packages
\usepackage{tikz}
\usetikzlibrary{automata,positioning}

\usepackage{listings}
\usepackage{xcolor}

\definecolor{dkgreen}{rgb}{0,0.6,0}
\definecolor{gray}{rgb}{0.5,0.5,0.5}
\definecolor{mauve}{rgb}{0.58,0,0.82}
%Additional Packages
\usepackage{listings}
\usepackage{subcaption}
\lstdefinestyle{myScalastyle}{
	frame=tb,
	language=scala,
	aboveskip=3mm,
	belowskip=3mm,
	showstringspaces=false,
	columns=flexible,
	basicstyle={\small\ttfamily},
	numbers=none,
	numberstyle=\tiny\color{gray},
	keywordstyle=\color{blue},
	commentstyle=\color{dkgreen},
	stringstyle=\color{mauve},
	frame=single,
	breaklines=true,
	breakatwhitespace=true,
	tabsize=3,
}
\begin{document}
	\maketitle
	\pagebreak
	\section{Abstract}
		Lorem ipsum dolor sit amet, consectetur adipiscing elit. Phasellus vitae bibendum risus. Vestibulum mattis dui eros, 
		eu tristique orci egestas eu. Maecenas hendrerit mi eget nulla malesuada hendrerit. Praesent egestas dui eget ipsum fringilla, 
		vitae sagittis urna varius. In hac habitasse platea dictumst. In nec magna tellus. Nullam tempus rutrum lectus, nec ornare urna 
		posuere non. Praesent nec arcu tristique nisi elementum luctus nec quis lorem. Cras ullamcorper urna vitae volutpat euismod. Nunc 
		tincidunt lorem et augue interdum sodales. Quisque erat mi, viverra ut quam a, rutrum ornare mi. Donec eget sagittis metus. 
		Vestibulum ante ipsum primis in faucibus orci. 
	\pagebreak
	\section{Continous Optimisation}
		\subsection{Task 1:Continous Code}
			\begin{note}
				Please find extensive documentation on how to run the following examples on /continous/README.md
			\end{note}
		\subsection{Subtask 1.C: Performance}
			For the evaluation of the algorithm, we will use two standarized functions, 
			\subsubsection{Sphrere}
			the Sphere(Commondly known as $F_{1}$ in the literature\cite{performance}) contains a single minima and its considered a easily solveable function.
			\begin{equation}
			f(x)=\sum_{i=1}^{n}{x_{i}^2}
			\end{equation}
			\subsubsection{Rastrigin's function}
			Rastrigin’s function(Commondly known as $F_{4}$ in the literature\cite{performance}), is considered a very difficult task due to its large number of local minima and its enormous search space.
			\begin{equation}
			f(x)=10\cdot n+\sum_{i=1}^{n}[x_{i}^{2}-10\cos(2\pi x_{i})]
			\end{equation}
			\subsubsection{Results}
			We will evaluate against, Mutation and Crossover rates as well as population. 
			The following results are the averages after 10 runs for each function, on 4 dimensions, using balanced selection and standard normal mutation distribution(see below).
			\begin{figure}[H]
				\centering
				\begin{subfigure}{.5\textwidth}
					\centering
					\begin{center}
						\begin{tabular}{||c c c||} 
							\hline
							MR & Avg F1 & Avg F4 \\ [0.5ex] 
							\hline\hline
							0.2 & 36 & 35 \\ 
							\hline
							0.4 & 96 & 104 \\
							\hline
							0.6 & 598 & 510 \\
							\hline
						\end{tabular}
					\end{center}
					\caption{Mutation rate (CR=0.8,Population size=10000)}
					\label{fig:sub1}
				\end{subfigure}%
				\begin{subfigure}{.5\textwidth}
					\centering
					\begin{center}
						\begin{tabular}{||c c c||} 
							\hline
							CR & Avg F1 & Avg F4 \\ [0.5ex] 
							\hline\hline
							0.2 & 125 & 127 \\ 
							\hline
							0.4 & 82 & 66 \\
							\hline
							0.6 & 53 & 42 \\
							\hline
							0.8 & 42 & 38 \\
							\hline
							1 & 30 & 29 \\
							\hline
						\end{tabular}
					\end{center}
					\caption{Crossover rate (MR=0.2,Population size=10000)}
					\label{fig:sub2}
				\end{subfigure}
				\begin{subfigure}{.5\textwidth}
					\centering
					\begin{center}
						\begin{tabular}{||c c c||} 
							\hline
							Population size & Avg F1 & Avg F4 \\ [0.5ex] 
							\hline\hline
							100 & 536 & 344 \\ 
							\hline
							1000 & 88 & 103 \\
							\hline
							10000 & 43 & 31 \\
							\hline
						\end{tabular}
					\end{center}
					\caption{Population size (MR=0.2,CR=0.8)}
					\label{fig:sub2}
				\end{subfigure}
				\caption{Various hyperparameters and their respective affect on performance}
				\label{fig:test}
			\end{figure}
			With the results in mind i can conclude the following 
			\begin{itemize}
				\item High mutation rate creates an oscillation effect around the minima, worsening the performance, 0.2 seems to be the best choice
				\item Low crossover rate worsens the performance, as it enforces the algorithm to multuple extra generations to converge into the minima.
				\item High population size seems to increase the performance\footnote{When the metric used is 'Number of generations'}
			\end{itemize}
		\subsection{Subtask 1.D, algorithm tuning}
			\begin{itemize}
				\item Mutation distribution
				\item Wheel and Elitistic selection
			\end{itemize}
			\subsubsection{Balanced and Elitistic selection}
				The initial implementation of the selection operator, Involved a simple wheel selection based on the relative fitness of the individual on the current generation.
				After some experimentation, this approach seemed 

		
		
		
\begin{thebibliography}{1}	
	\bibitem{wheel-selection}
	\textit{Bäck, Thomas, Evolutionary Algorithms in Theory and Practice (1996), p. 120, Oxford Univ. Press}
	
	\bibitem{holland-1975}
	\textit{Holland J.H. (1984) Genetic Algorithms and Adaptation. In: Selfridge O.G., Rissland E.L., Arbib M.A. (eds) Adaptive Control of Ill-Defined Systems. NATO Conference Series (II Systems Science), vol 16. Springer, Boston, MA. https://doi.org/10.1007/978-1-4684-8941-5\_21}
	
	\bibitem{performance}
	\textit{Carvalho, D. B. et al. “The Simple Genetic Algorithm Performance: A Comparative Study on the Operators Combination.” (2011).}
	
\end{thebibliography}
			
\end{document}