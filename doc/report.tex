\documentclass[openany]{article}


%Standard Stefanos Packages
\usepackage[utf8]{inputenc}
\usepackage{dirtytalk}
\usepackage{amsmath}
\usepackage{mathtools}  
\mathtoolsset{showonlyrefs} 
\usepackage{graphicx}
\usepackage{mdframed}
\usepackage{lipsum}
\usepackage{cancel}
\usepackage{systeme}
\usepackage{pgfplots}
\usepackage{textcomp}
\usepackage{amssymb}
\usepackage{geometry}
\usepackage{tikz-cd}
\usetikzlibrary{arrows}
\geometry{a4paper}
\graphicspath{ {./res/} }
\usepackage{float}
\restylefloat{table}
\newcommand{\comment}[1]{%
	\text{\phantom{(#1)}} \tag{#1}
}
 \title{\line(3,0){250}\\Artificial Intelligence \\ Genetic Algorithms Coursework  \\\line(3,0){250}}
\usepackage{pgfplots}
\newmdtheoremenv{note}{Note}
\pgfplotsset{compat=1.17}

\newmdtheoremenv{definition}{Definition}

%Extra Packages
\usepackage{tikz}
\usetikzlibrary{automata,positioning}

\usepackage{listings}
\usepackage{xcolor}

\definecolor{dkgreen}{rgb}{0,0.6,0}
\definecolor{gray}{rgb}{0.5,0.5,0.5}
\definecolor{mauve}{rgb}{0.58,0,0.82}
%Additional Packages
\usepackage{listings}
\usepackage{subcaption}
\lstdefinestyle{myScalastyle}{
	frame=tb,
	language=scala,
	aboveskip=3mm,
	belowskip=3mm,
	showstringspaces=false,
	columns=flexible,
	basicstyle={\small\ttfamily},
	numbers=none,
	numberstyle=\tiny\color{gray},
	keywordstyle=\color{blue},
	commentstyle=\color{dkgreen},
	stringstyle=\color{mauve},
	frame=single,
	breaklines=true,
	breakatwhitespace=true,
	tabsize=3,
}
\begin{document}
	\maketitle
	\pagebreak
	\section{Abstract}
		Lorem ipsum dolor sit amet, consectetur adipiscing elit. Phasellus vitae bibendum risus. Vestibulum mattis dui eros, 
		eu tristique orci egestas eu. Maecenas hendrerit mi eget nulla malesuada hendrerit. Praesent egestas dui eget ipsum fringilla, 
		vitae sagittis urna varius. In hac habitasse platea dictumst. In nec magna tellus. Nullam tempus rutrum lectus, nec ornare urna 
		posuere non. Praesent nec arcu tristique nisi elementum luctus nec quis lorem. Cras ullamcorper urna vitae volutpat euismod. Nunc 
		tincidunt lorem et augue interdum sodales. Quisque erat mi, viverra ut quam a, rutrum ornare mi. Donec eget sagittis metus. 
		Vestibulum ante ipsum primis in faucibus orci. 
	\pagebreak
	\section{Continous Optimisation}
		\begin{note}
			All the results for this section was produced with the following parameters
			\begin{itemize}
				\item N=4
				\item Lower bound = -5
				\item Upper bound = 5
			\end{itemize}
			No further mentions will be provided for those hyperparameters
		\end{note}
		\subsection{Subtask 1.C: Performance}
			For the evaluation of the algorithm, we will use two standarized functions, 
			\subsubsection{Sphrere}
				the Sphere(Commondly known as $F_{1}$ in the literature\cite{performance}) contains a single minima and its considered a easily solveable function.
				\begin{equation}
				f(x)=\sum_{i=1}^{n}{x_{i}^2}
				\end{equation}
			\subsubsection{Rastrigin's function}
				Rastrigin’s function(Commondly known as $F_{4}$ in the literature\cite{performance}), is considered a very difficult task due to its large number of local minima and its enormous search space.
				\begin{equation}
				f(x)=10\cdot n+\sum_{i=1}^{n}[x_{i}^{2}-10\cos(2\pi x_{i})]
				\end{equation}
			\subsubsection{Optimal hyperparameters}
				We will evaluate against, Mutation and Crossover rates as well as population. 
				The following results are the averages after 10 runs for each function, on 4 dimensions, using balanced selection and standard normal mutation distribution(see below).
			\subsubsection{Performance Results}

				\begin{note}
					The following results was produced with the following hyperparameters and operators
					\begin{itemize}
						\item Population 1000
						\item Crossover Rate 0.8
						\item Mutation Rate 0.2
						\item Elitistic Search(see next subsection)
						\item Uniform mutation step(see next subsection)
					\end{itemize}
				\end{note}
				
				
				
				
				
		\subsection{Subtask 1.D, algorithm tuning}
			\begin{itemize}
				\item Balanced and Elitistic Wheel selection
				\item Mutation distribution
			\end{itemize}
			\subsubsection{Balanced and Elitistic selection}
				The initial implementation of the selection operator, involved a balanced wheel selection based on the relative fitness of the individual on the current generation, as well as
				an individual probability of each individual to mate. Using both metrics we determined if an individual was allowed to mate or not. This implementation allowed less ideal
				candidates to mate with some probability, as this approach allowed the algorithm to avoid local minima traps easily. After some refinement this approach was ambandoned for a 
				simpler elitistic wheel selection. In this selection strategy, the relative fitness is the only factor that affects an individuals probability to mate. The data below shows the 
				improvement using the simpler method.
			\subsubsection{Gaussian and Uniform mutation step}
			The initial implementation of the mutation operator used a step taken from a Gaussian distribution with $\mu=0,\sigma=1$. The desired effect was to select a tiny step as a mutation operation the 
			majority of the times, while leaving a small probability for a larger step, statistically this was used to avoid again, local minima. After some experimentation we improved our 
			mutation operator by choosing a uniform distribution within the range [-1,1]. As the probability for a bigger step increases, we gain a significant performance boost when we have a relatively small
			mutation probability.
				\begin{note}
					The following results was produced with the following hyperparameters
					\begin{itemize}
						\item Population 1000
						\item Crossover Rate 0.8
						\item Mutation Rate 0.2
					\end{itemize}
				\end{note}
				\begin{figure}[H]
					\centering
					\begin{subfigure}{.5\textwidth}
						\centering
						\begin{center}
							\begin{tabular}{||c c c||} 
								\hline
								  & Balanced & Elitistic \\ [0.5ex] 
								\hline\hline
								F1 & 124 & 98 \\ 
								\hline
								F4 & 107 & 87 \\
								\hline
							\end{tabular}
						\end{center}
						\caption{Balanced vs Elitistic(Gaussian Mutation)}
						\label{fig:sub1}
					\end{subfigure}%
					\begin{subfigure}{.5\textwidth}
						\centering
						\begin{center}
							\begin{tabular}{||c c c||} 
								\hline
								& Balanced & Elitistic \\ [0.5ex] 
								\hline\hline
								F1 & 133 & 90 \\ 
								\hline
								F4 & 94 & 67 \\
								\hline
							\end{tabular}
						\end{center}
						\caption{Balanced vs Elitistic(Uniform Mutation)}
						\label{fig:sub2}
					\end{subfigure}
					\caption{A}
					\label{fig:test}
				\end{figure}
				

		
		
		
\begin{thebibliography}{1}	
	\bibitem{wheel-selection}
	\textit{Bäck, Thomas, Evolutionary Algorithms in Theory and Practice (1996), p. 120, Oxford Univ. Press}
	
	\bibitem{holland-1975}
	\textit{Holland J.H. (1984) Genetic Algorithms and Adaptation. In: Selfridge O.G., Rissland E.L., Arbib M.A. (eds) Adaptive Control of Ill-Defined Systems. NATO Conference Series (II Systems Science), vol 16. Springer, Boston, MA. https://doi.org/10.1007/978-1-4684-8941-5\_21}
	
	\bibitem{performance}
	\textit{Carvalho, D. B. et al. “The Simple Genetic Algorithm Performance: A Comparative Study on the Operators Combination.” (2011).}
	
\end{thebibliography}
			
\end{document}